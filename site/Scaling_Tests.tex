% Options for packages loaded elsewhere
% Options for packages loaded elsewhere
\PassOptionsToPackage{unicode}{hyperref}
\PassOptionsToPackage{hyphens}{url}
\PassOptionsToPackage{dvipsnames,svgnames,x11names}{xcolor}
%
\documentclass[
]{report}
\usepackage{xcolor}
\usepackage[left=20.0mm,right=20.0mm,marginparsep=7.7mm,marginparwidth=70.3mm,top=20mm]{geometry}
\usepackage{amsmath,amssymb}
\setcounter{secnumdepth}{5}
\usepackage{iftex}
\ifPDFTeX
  \usepackage[T1]{fontenc}
  \usepackage[utf8]{inputenc}
  \usepackage{textcomp} % provide euro and other symbols
\else % if luatex or xetex
  \usepackage{unicode-math} % this also loads fontspec
  \defaultfontfeatures{Scale=MatchLowercase}
  \defaultfontfeatures[\rmfamily]{Ligatures=TeX,Scale=1}
\fi
\usepackage{lmodern}
\ifPDFTeX\else
  % xetex/luatex font selection
\fi
% Use upquote if available, for straight quotes in verbatim environments
\IfFileExists{upquote.sty}{\usepackage{upquote}}{}
\IfFileExists{microtype.sty}{% use microtype if available
  \usepackage[]{microtype}
  \UseMicrotypeSet[protrusion]{basicmath} % disable protrusion for tt fonts
}{}
\makeatletter
\@ifundefined{KOMAClassName}{% if non-KOMA class
  \IfFileExists{parskip.sty}{%
    \usepackage{parskip}
  }{% else
    \setlength{\parindent}{0pt}
    \setlength{\parskip}{6pt plus 2pt minus 1pt}}
}{% if KOMA class
  \KOMAoptions{parskip=half}}
\makeatother
% Make \paragraph and \subparagraph free-standing
\makeatletter
\ifx\paragraph\undefined\else
  \let\oldparagraph\paragraph
  \renewcommand{\paragraph}{
    \@ifstar
      \xxxParagraphStar
      \xxxParagraphNoStar
  }
  \newcommand{\xxxParagraphStar}[1]{\oldparagraph*{#1}\mbox{}}
  \newcommand{\xxxParagraphNoStar}[1]{\oldparagraph{#1}\mbox{}}
\fi
\ifx\subparagraph\undefined\else
  \let\oldsubparagraph\subparagraph
  \renewcommand{\subparagraph}{
    \@ifstar
      \xxxSubParagraphStar
      \xxxSubParagraphNoStar
  }
  \newcommand{\xxxSubParagraphStar}[1]{\oldsubparagraph*{#1}\mbox{}}
  \newcommand{\xxxSubParagraphNoStar}[1]{\oldsubparagraph{#1}\mbox{}}
\fi
\makeatother

\usepackage{color}
\usepackage{fancyvrb}
\newcommand{\VerbBar}{|}
\newcommand{\VERB}{\Verb[commandchars=\\\{\}]}
\DefineVerbatimEnvironment{Highlighting}{Verbatim}{commandchars=\\\{\}}
% Add ',fontsize=\small' for more characters per line
\usepackage{framed}
\definecolor{shadecolor}{RGB}{241,243,245}
\newenvironment{Shaded}{\begin{snugshade}}{\end{snugshade}}
\newcommand{\AlertTok}[1]{\textcolor[rgb]{0.68,0.00,0.00}{#1}}
\newcommand{\AnnotationTok}[1]{\textcolor[rgb]{0.37,0.37,0.37}{#1}}
\newcommand{\AttributeTok}[1]{\textcolor[rgb]{0.40,0.45,0.13}{#1}}
\newcommand{\BaseNTok}[1]{\textcolor[rgb]{0.68,0.00,0.00}{#1}}
\newcommand{\BuiltInTok}[1]{\textcolor[rgb]{0.00,0.23,0.31}{#1}}
\newcommand{\CharTok}[1]{\textcolor[rgb]{0.13,0.47,0.30}{#1}}
\newcommand{\CommentTok}[1]{\textcolor[rgb]{0.37,0.37,0.37}{#1}}
\newcommand{\CommentVarTok}[1]{\textcolor[rgb]{0.37,0.37,0.37}{\textit{#1}}}
\newcommand{\ConstantTok}[1]{\textcolor[rgb]{0.56,0.35,0.01}{#1}}
\newcommand{\ControlFlowTok}[1]{\textcolor[rgb]{0.00,0.23,0.31}{\textbf{#1}}}
\newcommand{\DataTypeTok}[1]{\textcolor[rgb]{0.68,0.00,0.00}{#1}}
\newcommand{\DecValTok}[1]{\textcolor[rgb]{0.68,0.00,0.00}{#1}}
\newcommand{\DocumentationTok}[1]{\textcolor[rgb]{0.37,0.37,0.37}{\textit{#1}}}
\newcommand{\ErrorTok}[1]{\textcolor[rgb]{0.68,0.00,0.00}{#1}}
\newcommand{\ExtensionTok}[1]{\textcolor[rgb]{0.00,0.23,0.31}{#1}}
\newcommand{\FloatTok}[1]{\textcolor[rgb]{0.68,0.00,0.00}{#1}}
\newcommand{\FunctionTok}[1]{\textcolor[rgb]{0.28,0.35,0.67}{#1}}
\newcommand{\ImportTok}[1]{\textcolor[rgb]{0.00,0.46,0.62}{#1}}
\newcommand{\InformationTok}[1]{\textcolor[rgb]{0.37,0.37,0.37}{#1}}
\newcommand{\KeywordTok}[1]{\textcolor[rgb]{0.00,0.23,0.31}{\textbf{#1}}}
\newcommand{\NormalTok}[1]{\textcolor[rgb]{0.00,0.23,0.31}{#1}}
\newcommand{\OperatorTok}[1]{\textcolor[rgb]{0.37,0.37,0.37}{#1}}
\newcommand{\OtherTok}[1]{\textcolor[rgb]{0.00,0.23,0.31}{#1}}
\newcommand{\PreprocessorTok}[1]{\textcolor[rgb]{0.68,0.00,0.00}{#1}}
\newcommand{\RegionMarkerTok}[1]{\textcolor[rgb]{0.00,0.23,0.31}{#1}}
\newcommand{\SpecialCharTok}[1]{\textcolor[rgb]{0.37,0.37,0.37}{#1}}
\newcommand{\SpecialStringTok}[1]{\textcolor[rgb]{0.13,0.47,0.30}{#1}}
\newcommand{\StringTok}[1]{\textcolor[rgb]{0.13,0.47,0.30}{#1}}
\newcommand{\VariableTok}[1]{\textcolor[rgb]{0.07,0.07,0.07}{#1}}
\newcommand{\VerbatimStringTok}[1]{\textcolor[rgb]{0.13,0.47,0.30}{#1}}
\newcommand{\WarningTok}[1]{\textcolor[rgb]{0.37,0.37,0.37}{\textit{#1}}}

\usepackage{longtable,booktabs,array}
\usepackage{calc} % for calculating minipage widths
% Correct order of tables after \paragraph or \subparagraph
\usepackage{etoolbox}
\makeatletter
\patchcmd\longtable{\par}{\if@noskipsec\mbox{}\fi\par}{}{}
\makeatother
% Allow footnotes in longtable head/foot
\IfFileExists{footnotehyper.sty}{\usepackage{footnotehyper}}{\usepackage{footnote}}
\makesavenoteenv{longtable}
\usepackage{graphicx}
\makeatletter
\newsavebox\pandoc@box
\newcommand*\pandocbounded[1]{% scales image to fit in text height/width
  \sbox\pandoc@box{#1}%
  \Gscale@div\@tempa{\textheight}{\dimexpr\ht\pandoc@box+\dp\pandoc@box\relax}%
  \Gscale@div\@tempb{\linewidth}{\wd\pandoc@box}%
  \ifdim\@tempb\p@<\@tempa\p@\let\@tempa\@tempb\fi% select the smaller of both
  \ifdim\@tempa\p@<\p@\scalebox{\@tempa}{\usebox\pandoc@box}%
  \else\usebox{\pandoc@box}%
  \fi%
}
% Set default figure placement to htbp
\def\fps@figure{htbp}
\makeatother





\setlength{\emergencystretch}{3em} % prevent overfull lines

\providecommand{\tightlist}{%
  \setlength{\itemsep}{0pt}\setlength{\parskip}{0pt}}



 


\usepackage{unicode-math}
\DeclareMathOperator{\var}{\mathbb{V}\mathrm{ar}}
\DeclareMathOperator{\cov}{\mathbb{C}\mathrm{ov}}
\newcommand\eqc{\stackrel{c}{=}}
\newcommand{\bv}[1]{\symbfit{#1}}
\setmainfont{XITS}
\setmathfont{XITS Math}
\usepackage[framemethod=tikz]{mdframed}

% Define custom mdframed settings for shading
\mdfdefinestyle{thmstyle}{
  skipabove=12,skipbelow=12pt,
  linecolor=black,
  linewidth=0.5pt,
  backgroundcolor=gray!10,
  innerleftmargin=10pt,
  innerrightmargin=10pt,
  innertopmargin=10pt,
  innerbottommargin=10pt,
  roundcorner=5pt
}

% Redefine theorem and lemma environments to use mdframed
\surroundwithmdframed[style=thmstyle]{theorem}
\surroundwithmdframed[style=thmstyle]{lemma}
\surroundwithmdframed[style=thmstyle]{proof}
\surroundwithmdframed[style=thmstyle]{Shaded}
\makeatletter
\@ifpackageloaded{caption}{}{\usepackage{caption}}
\AtBeginDocument{%
\ifdefined\contentsname
  \renewcommand*\contentsname{Table of contents}
\else
  \newcommand\contentsname{Table of contents}
\fi
\ifdefined\listfigurename
  \renewcommand*\listfigurename{List of Figures}
\else
  \newcommand\listfigurename{List of Figures}
\fi
\ifdefined\listtablename
  \renewcommand*\listtablename{List of Tables}
\else
  \newcommand\listtablename{List of Tables}
\fi
\ifdefined\figurename
  \renewcommand*\figurename{Figure}
\else
  \newcommand\figurename{Figure}
\fi
\ifdefined\tablename
  \renewcommand*\tablename{Table}
\else
  \newcommand\tablename{Table}
\fi
}
\@ifpackageloaded{float}{}{\usepackage{float}}
\floatstyle{ruled}
\@ifundefined{c@chapter}{\newfloat{codelisting}{h}{lop}}{\newfloat{codelisting}{h}{lop}[chapter]}
\floatname{codelisting}{Listing}
\newcommand*\listoflistings{\listof{codelisting}{List of Listings}}
\makeatother
\makeatletter
\makeatother
\makeatletter
\@ifpackageloaded{caption}{}{\usepackage{caption}}
\@ifpackageloaded{subcaption}{}{\usepackage{subcaption}}
\makeatother
\usepackage{bookmark}
\IfFileExists{xurl.sty}{\usepackage{xurl}}{} % add URL line breaks if available
\urlstyle{same}
\hypersetup{
  pdftitle={Scaling Comparisons for the filtering algorithms},
  pdfauthor={Evan Tate Paterson Hughes},
  colorlinks=true,
  linkcolor={blue},
  filecolor={Maroon},
  citecolor={Blue},
  urlcolor={Blue},
  pdfcreator={LaTeX via pandoc}}


\title{Scaling Comparisons for the filtering algorithms}
\author{Evan Tate Paterson Hughes}
\date{}
\begin{document}
\maketitle

\renewcommand*\contentsname{Table of contents}
{
\hypersetup{linkcolor=}
\setcounter{tocdepth}{2}
\tableofcontents
}

Currently, the Kalman filter, the Information filter, the Square-root
filter, and the square-root-information filter are all implemented.

We will use a simulated data set, from cosine basis functions over 100
frequencies and invariant kernel. See figure (?) for the 6 time points
of the simulated process.

We then filter this data, and time the implemented filters to compute
the marginal log likelihood. We run the alogrithms with a sequence of
model parameters that are close to the ones used to simulate the data,
with added noise to avoid the memoisation the JAX uses.

We simulate \(n=300\) observations per time point in total, and redact
observations in order to show the compute times for each \(n\) between
50 and 300. These are computed in 64 bit for stability, and the results
are in figure (?)

Additionally, to demonstrate the square root filters stability, we also
fix \(n\) and progressively increase \(n\). The data is simulated from a
high-dimensional process basis, and then filtered using models with a
lower \(r\), increasing in square between \(5^2\) and \(15^2\). The
results are shown in figure (?).

\begin{Shaded}
\begin{Highlighting}[]
\ImportTok{import}\NormalTok{ pandas }\ImportTok{as}\NormalTok{ pd}
\ImportTok{import}\NormalTok{ plotly.express }\ImportTok{as}\NormalTok{ px}

\CommentTok{\# Load the datasets}
\NormalTok{df64 }\OperatorTok{=}\NormalTok{ pd.read\_csv(}\StringTok{\textquotesingle{}data/varying\_n\_laptop\_64.csv\textquotesingle{}}\NormalTok{)}
\NormalTok{df32 }\OperatorTok{=}\NormalTok{ pd.read\_csv(}\StringTok{\textquotesingle{}data/varying\_n\_laptop\_32.csv\textquotesingle{}}\NormalTok{)}

\CommentTok{\# Select and melt relevant columns from df64}
\NormalTok{df64\_melted }\OperatorTok{=}\NormalTok{ df64.melt(}
\NormalTok{    id\_vars}\OperatorTok{=}\StringTok{"n"}\NormalTok{,}
\NormalTok{    value\_vars}\OperatorTok{=}\NormalTok{df64.columns[}\DecValTok{3}\NormalTok{:}\DecValTok{7}\NormalTok{],}
\NormalTok{    var\_name}\OperatorTok{=}\StringTok{"Filter"}\NormalTok{,}
\NormalTok{    value\_name}\OperatorTok{=}\StringTok{"Time"}
\NormalTok{)}
\NormalTok{df64\_melted[}\StringTok{"Precision"}\NormalTok{] }\OperatorTok{=} \StringTok{"64{-}bit"}

\CommentTok{\# Select and melt the two columns from df32}
\NormalTok{df32\_melted }\OperatorTok{=}\NormalTok{ df32.melt(}
\NormalTok{    id\_vars}\OperatorTok{=}\StringTok{"n"}\NormalTok{,}
\NormalTok{    value\_vars}\OperatorTok{=}\NormalTok{[df32.columns[}\DecValTok{4}\NormalTok{], df32.columns[}\DecValTok{6}\NormalTok{]],}
\NormalTok{    var\_name}\OperatorTok{=}\StringTok{"Filter"}\NormalTok{,}
\NormalTok{    value\_name}\OperatorTok{=}\StringTok{"Time"}
\NormalTok{)}
\NormalTok{df32\_melted[}\StringTok{"Precision"}\NormalTok{] }\OperatorTok{=} \StringTok{"32{-}bit"}

\CommentTok{\# Combine both}
\NormalTok{df\_combined }\OperatorTok{=}\NormalTok{ pd.concat([df64\_melted, df32\_melted], ignore\_index}\OperatorTok{=}\VariableTok{True}\NormalTok{)}

\CommentTok{\# Plot using Plotly Express}
\NormalTok{fig }\OperatorTok{=}\NormalTok{ px.line(}
\NormalTok{    df\_combined,}
\NormalTok{    x}\OperatorTok{=}\StringTok{"n"}\NormalTok{,}
\NormalTok{    y}\OperatorTok{=}\StringTok{"Time"}\NormalTok{,}
\NormalTok{    color}\OperatorTok{=}\StringTok{"Filter"}\NormalTok{,}
\NormalTok{    line\_dash}\OperatorTok{=}\StringTok{"Precision"}\NormalTok{,}
\NormalTok{    labels}\OperatorTok{=}\NormalTok{\{}\StringTok{"n"}\NormalTok{: }\StringTok{"number of observations"}\NormalTok{, }\StringTok{"Time"}\NormalTok{: }\StringTok{"Time (seconds)"}\NormalTok{\},}
\NormalTok{    title}\OperatorTok{=}\StringTok{"Filters Comparison with Varying n"}
\NormalTok{)}

\NormalTok{fig.write\_image(}\StringTok{"figure/scale\_test\_n.png"}\NormalTok{)}
\NormalTok{fig.show()}
\end{Highlighting}
\end{Shaded}

\begin{verbatim}
Unable to display output for mime type(s): text/html
\end{verbatim}

\begin{verbatim}
Unable to display output for mime type(s): text/html
\end{verbatim}

\begin{Shaded}
\begin{Highlighting}[]
\ImportTok{import}\NormalTok{ jax}
\ImportTok{import}\NormalTok{ jax.numpy }\ImportTok{as}\NormalTok{ jnp}
\ImportTok{import}\NormalTok{ pandas }\ImportTok{as}\NormalTok{ pd}
\ImportTok{import}\NormalTok{ plotly.express }\ImportTok{as}\NormalTok{ px}

\NormalTok{df }\OperatorTok{=}\NormalTok{ pd.read\_csv(}\StringTok{\textquotesingle{}data/varying\_n\_64.csv\textquotesingle{}}\NormalTok{)}

\NormalTok{fig }\OperatorTok{=}\NormalTok{ px.line(}
\NormalTok{    df, }
\NormalTok{    x}\OperatorTok{=}\StringTok{"n"}\NormalTok{,}
\NormalTok{    y}\OperatorTok{=}\NormalTok{df.columns[}\DecValTok{3}\NormalTok{:}\DecValTok{7}\NormalTok{], }
\NormalTok{    labels}\OperatorTok{=}\NormalTok{\{}\StringTok{"n"}\NormalTok{: }\StringTok{"number of observations"}\NormalTok{, }\StringTok{"value"}\NormalTok{: }\StringTok{"Time (seconds)"}\NormalTok{\},}
\NormalTok{    title}\OperatorTok{=}\StringTok{"Average compute time Filters with Varying n (64 bit)"}
\NormalTok{)}

\CommentTok{\# Show the plot}
\NormalTok{fig.write\_image(}\StringTok{"figure/scale\_test\_n\_64.png"}\NormalTok{)}
\NormalTok{fig.show()}
\end{Highlighting}
\end{Shaded}

\begin{verbatim}
Unable to display output for mime type(s): text/html
\end{verbatim}

\begin{Shaded}
\begin{Highlighting}[]
\ImportTok{import}\NormalTok{ jax}
\ImportTok{import}\NormalTok{ jax.numpy }\ImportTok{as}\NormalTok{ jnp}
\ImportTok{import}\NormalTok{ pandas }\ImportTok{as}\NormalTok{ pd}
\ImportTok{import}\NormalTok{ plotly.express }\ImportTok{as}\NormalTok{ px}

\NormalTok{df }\OperatorTok{=}\NormalTok{ pd.read\_csv(}\StringTok{\textquotesingle{}data/varying\_r\_32.csv\textquotesingle{}}\NormalTok{)}

\NormalTok{fig }\OperatorTok{=}\NormalTok{ px.line(}
\NormalTok{    df, }
\NormalTok{    x}\OperatorTok{=}\StringTok{"r"}\NormalTok{,}
\NormalTok{    y}\OperatorTok{=}\NormalTok{df.columns[}\DecValTok{7}\NormalTok{:], }
\NormalTok{    labels}\OperatorTok{=}\NormalTok{\{}\StringTok{"n"}\NormalTok{: }\StringTok{"number of observations"}\NormalTok{, }\StringTok{"value"}\NormalTok{: }\StringTok{"Time (seconds)"}\NormalTok{\},}
\NormalTok{    title}\OperatorTok{=}\StringTok{"Compte Time of Filters Varying r (32 bit)"}
\NormalTok{)}

\CommentTok{\# Show the plot}
\NormalTok{fig.write\_image(}\StringTok{"figure/scale\_test\_r\_32.png"}\NormalTok{)}
\NormalTok{fig.show()}
\end{Highlighting}
\end{Shaded}

\begin{verbatim}
Unable to display output for mime type(s): text/html
\end{verbatim}




\end{document}
